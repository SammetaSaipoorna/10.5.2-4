\documentclass[12pt]{article}
\usepackage{amsmath}  % for align* environment
\usepackage{array}    % for creating tables

\title{Discrete Mathematics}
\author{Samméta Saipoorna\\ EE23BTECH11055}
\date{}

\begin{document}
\maketitle

\section{Arithmetic Progression Problem}
Which term of the arithmetic progression (AP): $3, 8, 13, 18, \ldots$, is 78?

\begin{table}[h]
    \centering
    \begin{tabular}{|c|c|c|}
        \hline
        \textbf{Parameter} & \textbf{Value} & \textbf{Description} \\
        \hline
        $\mathrm{x}(0)$ & 3 & Initial term in the AP \\
        \hline
        $\mathrm{d}$ & 5 & Difference between consecutive terms \\
        \hline
        $\mathrm{x}(k) \times \mathrm{u}(k)$ & 78 & Target term in the AP \\
        \hline
    \end{tabular}
    \caption{Input Parameters for Arithmetic Progression}
    \label{tab:input_parameters}
\end{table}

To find the term number ($k$) when the $k$-th term $\mathrm{x}(k)$ is 78 in the arithmetic progression (AP) with the given input parameters, we can use the formula:
\begin{align}
 \mathrm{x}(n) &= [\mathrm{x}(0) + (n-1)\mathrm{d}] \times \mathrm{u}(n) \label{eq:ap_formula} \\
\end{align}
Substituting the values from Table~\ref{tab:input_parameters} into Equation~\eqref{eq:ap_formula}:
\begin{align*}
 78 &= 3 + (k-1) \times 5 \\
 75 &= 5(k-1) \\
 15 &= k-1 \\
 k &= 16 \\
\end{align*}

Therefore, the term number ($k$) when $\mathrm{x}(k) = 78$ in the given arithmetic progression is 16.

\end{document}
