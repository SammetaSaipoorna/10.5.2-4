\documentclass{article}
\usepackage{amsmath}  % for align* environment
\usepackage{array}    % for creating tables
\title {DISCRETE MATHEMATICS}
\author {SAMMETA SAIPOORNA\\
               EE23BTECH11055}
\begin{document}
Which term of the AP: 3, 8, 13, 18, \ldots, is 78?
\begin{table}[h]
    \centering
    \begin{tabular}{|c|c|}
        \hline
        \textbf{Parameter} & \textbf{Value} \\
        \hline
        First term (\( a_1 \)) & 3 \\
        Common difference (\( d \)) & 5 \\
        \(n\)-th term (\( a_n \)) & 78 \\
        \hline
    \end{tabular}
    \caption{Input Parameters for Arithmetic Progression}
    \label{tab:input_parameters}
\end{table}


To find the term number (\( n \)) when the \(n\)-th term (\( a_n \)) is 78 in the arithmetic progression (AP) with the given input parameters, we can use the formula:

\[ a_n = a_1 + (n-1)d \]

Substituting the values from Table~\ref{tab:input_parameters}:

\[ 78 = 3 + (n-1) \times 5 \]

Now, solve for \( n \):

\begin{align*}
78 &= 3 + 5(n-1) \\
75 &= 5(n-1) \\
15 &= n-1 \\
n &= 16
\end{align*}

Therefore, the term number (\( n \)) when \( a_n = 78 \) in the given arithmetic progression is 16.

\end{document}
