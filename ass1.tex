\documentclass{article}
\usepackage{amsmath}  % for align* environment
\usepackage{array}    % for creating tables

\title{Discrete Mathematics}
\author{Samméta Saipoorna\\
        EE23BTECH11055}

\begin{document}
\maketitle

\section{Arithmetic Progression Problem}
Which term of the arithmetic progression (AP): $3, 8, 13, 18, \ldots$, is 78?

\begin{table}[h]
    \centering
    \begin{tabular}{|c|c|c|}
        \hline
        \textbf{Parameter} & \textbf{Value} & \textbf{Description} \\
        \hline
        $x(0)$ & 3 & Initial term in the AP \\
        $d$ & 5 & Difference between consecutive terms \\
        $x(n) \cdot u(n)$ & 78 & Target term in the AP \\
        \hline
    \end{tabular}
    \caption{Input Parameters for Arithmetic Progression}
    \label{tab:input_parameters}
\end{table}

To find the term number ($n$) when the $n$-th term $x(n)$ is 78 in the arithmetic progression (AP) with the given input parameters, we can use the formula:
\begin{align*}
 x(n) &= [x(0) + (n-1)d] \cdot u(n)
\end{align*}
Substituting the values from Table~\ref{tab:input_parameters}:
\begin{align*}
 78 &= 3 + (n-1) \cdot 5 \tag{1}
\end{align*}
Now, solve for $n$:
\begin{align*}
75 &= 5(n-1) \\
15 &= n-1 \\
n &= 16
\end{align*}

Therefore, the term number ($n$) when $x(n) = 78$ in the given arithmetic progression is 16.

\end{document}
