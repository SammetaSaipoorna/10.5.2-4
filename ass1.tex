\documentclass[12pt]{article}
\usepackage{amsmath}  % for align* environment

\title{Arithmetic Progression Problem}
\author{SAMMETA SAIPOORNA}
\date{}

% Define variables for the arithmetic progression
\newcommand{\initialterm}{3}
\newcommand{\commondifference}{5}
\newcommand{\targetterm}{78}

\begin{document}
\maketitle

\section*{Problem Statement}
Which term of the arithmetic progression (AP): \(3, 8, 13, 18, \ldots\) is \(78\)? 

Find the term (\(k\)) when the term \(x(k)\) is equal to 78.

\section{Input Table}
Here is the input table with common difference, initial term, and a description:

\begin{center}
\begin{tabular}{|c|c|c|}
  \hline
  \(S.n\) & Parameters & Description \\
  \hline
  1 & \(a(0)\) & Initial Term of the AP\\
  \hline
  2 & \(d\) & Common Difference \\
  \hline
  3 & \(a(k) \times u(k)\) & Target Term of the AP\\
  \hline
\end{tabular}
\end{center}

\section{Solution}
Let's solve the problem:
\begin{align*}
\text{Let } x(n) &= [\initialterm + (n-1)\commondifference] \times u(n) \\
\text{Given: } x(k) &= \targetterm \\
\text{Substitute values into the formula:} \\
\targetterm &= \initialterm + (k-1) \times \commondifference \\
\targetterm &= 3 + (k-1) \times 5 \\
\targetterm - 3 &= 5(k-1) \\
75 &= 5(k-1) \\
15 &= k-1 \\
k &= 16
\end{align*}
Therefore, the correct term number (\(k\)) when \(x(k) = 78\) in the given 

arithmetic progression is \(k = 16\).

\section{Z-Transform}
Let the Z-transform of \(x(n)\) be \(X(z)\). Let \(U(z)\) be the Z-transform of \(u(n)\).
\begin{align}
X(z) &= x(0)U(z) + dz^{-1}Z\{nu(n)\} \\
&= \frac{\initialterm}{1 - z^{-1}} + \frac{\commondifference z^{-1}}{(1 - z^{-1})^2} \quad \forall \quad |z| > 1
\end{align}

Using the values from the arithmetic progression problem:
\begin{align}
X(z) &= \frac{3}{1 - z^{-1}} + \frac{5z^{-1}}{(1 - z^{-1})^2} \quad \forall \quad |z| > 1
\end{align}

\end{document}
