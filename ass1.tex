\documentclass{article}
\usepackage{amsmath}  % for align* environment
\usepackage{array}    % for creating tables

\title{DISCRETE MATHEMATICS}
\author{SAMMETA SAIPOORNA\\
               EE23BTECH11055}

\begin{document}
\maketitle

\section{Arithmetic Progression Problem}
Which term of the arithmetic progression (AP): 3, 8, 13, 18, \ldots, is 78?

\begin{table}[h]
    \centering
    \begin{tabular}{|c|c|c|}
        \hline
        \textbf{Parameter} & \textbf{Value} & \textbf{Description} \\
        \hline
        First term (\( x_0 \)) & 3 & Initial term in the AP \\
        Common difference (\( d \)) & 5 & Difference between consecutive terms \\
        \(n\)-th term (\( x_n \)) & 78 & Target term in the AP \\
        \hline
    \end{tabular}
    \caption{Input Parameters for Arithmetic Progression}
    \label{tab:input_parameters}
\end{table}

To find the term number (\( n \)) when the \(n\)-th term (\( x_n \)) is 78 in the arithmetic progression (AP) with the given input parameters, we can use the formula:

\[ x_n = x_0 + (n-1)d \]

Substituting the values from Table~\ref{tab:input_parameters}:

\[ 78 = 3 + (n-1) \times 5 \]

Now, solve for \( n \):

\begin{align*}
78 &= 3 + 5(n-1) \label{eq:ap_equation} \\
75 &= 5(n-1) \\
15 &= n-1 \\
n &= 16
\end{align*}

Therefore, the term number (\( n \)) when \( x_n = 78 \) in the given arithmetic progression is 16.

\end{document}
